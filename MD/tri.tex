\documentclass[]{article}
\usepackage{lmodern}
\usepackage{amssymb,amsmath}
\usepackage{ifxetex,ifluatex}
\usepackage{fixltx2e} % provides \textsubscript
\ifnum 0\ifxetex 1\fi\ifluatex 1\fi=0 % if pdftex
  \usepackage[T1]{fontenc}
  \usepackage[utf8]{inputenc}
\else % if luatex or xelatex
  \ifxetex
    \usepackage{mathspec}
  \else
    \usepackage{fontspec}
  \fi
  \defaultfontfeatures{Ligatures=TeX,Scale=MatchLowercase}
\fi
% use upquote if available, for straight quotes in verbatim environments
\IfFileExists{upquote.sty}{\usepackage{upquote}}{}
% use microtype if available
\IfFileExists{microtype.sty}{%
\usepackage{microtype}
\UseMicrotypeSet[protrusion]{basicmath} % disable protrusion for tt fonts
}{}
\usepackage[margin=1in]{geometry}
\usepackage{hyperref}
\hypersetup{unicode=true,
            pdftitle={Teoria de Resposta ao Item: E-Class},
            pdfborder={0 0 0},
            breaklinks=true}
\urlstyle{same}  % don't use monospace font for urls
\usepackage{color}
\usepackage{fancyvrb}
\newcommand{\VerbBar}{|}
\newcommand{\VERB}{\Verb[commandchars=\\\{\}]}
\DefineVerbatimEnvironment{Highlighting}{Verbatim}{commandchars=\\\{\}}
% Add ',fontsize=\small' for more characters per line
\usepackage{framed}
\definecolor{shadecolor}{RGB}{248,248,248}
\newenvironment{Shaded}{\begin{snugshade}}{\end{snugshade}}
\newcommand{\KeywordTok}[1]{\textcolor[rgb]{0.13,0.29,0.53}{\textbf{#1}}}
\newcommand{\DataTypeTok}[1]{\textcolor[rgb]{0.13,0.29,0.53}{#1}}
\newcommand{\DecValTok}[1]{\textcolor[rgb]{0.00,0.00,0.81}{#1}}
\newcommand{\BaseNTok}[1]{\textcolor[rgb]{0.00,0.00,0.81}{#1}}
\newcommand{\FloatTok}[1]{\textcolor[rgb]{0.00,0.00,0.81}{#1}}
\newcommand{\ConstantTok}[1]{\textcolor[rgb]{0.00,0.00,0.00}{#1}}
\newcommand{\CharTok}[1]{\textcolor[rgb]{0.31,0.60,0.02}{#1}}
\newcommand{\SpecialCharTok}[1]{\textcolor[rgb]{0.00,0.00,0.00}{#1}}
\newcommand{\StringTok}[1]{\textcolor[rgb]{0.31,0.60,0.02}{#1}}
\newcommand{\VerbatimStringTok}[1]{\textcolor[rgb]{0.31,0.60,0.02}{#1}}
\newcommand{\SpecialStringTok}[1]{\textcolor[rgb]{0.31,0.60,0.02}{#1}}
\newcommand{\ImportTok}[1]{#1}
\newcommand{\CommentTok}[1]{\textcolor[rgb]{0.56,0.35,0.01}{\textit{#1}}}
\newcommand{\DocumentationTok}[1]{\textcolor[rgb]{0.56,0.35,0.01}{\textbf{\textit{#1}}}}
\newcommand{\AnnotationTok}[1]{\textcolor[rgb]{0.56,0.35,0.01}{\textbf{\textit{#1}}}}
\newcommand{\CommentVarTok}[1]{\textcolor[rgb]{0.56,0.35,0.01}{\textbf{\textit{#1}}}}
\newcommand{\OtherTok}[1]{\textcolor[rgb]{0.56,0.35,0.01}{#1}}
\newcommand{\FunctionTok}[1]{\textcolor[rgb]{0.00,0.00,0.00}{#1}}
\newcommand{\VariableTok}[1]{\textcolor[rgb]{0.00,0.00,0.00}{#1}}
\newcommand{\ControlFlowTok}[1]{\textcolor[rgb]{0.13,0.29,0.53}{\textbf{#1}}}
\newcommand{\OperatorTok}[1]{\textcolor[rgb]{0.81,0.36,0.00}{\textbf{#1}}}
\newcommand{\BuiltInTok}[1]{#1}
\newcommand{\ExtensionTok}[1]{#1}
\newcommand{\PreprocessorTok}[1]{\textcolor[rgb]{0.56,0.35,0.01}{\textit{#1}}}
\newcommand{\AttributeTok}[1]{\textcolor[rgb]{0.77,0.63,0.00}{#1}}
\newcommand{\RegionMarkerTok}[1]{#1}
\newcommand{\InformationTok}[1]{\textcolor[rgb]{0.56,0.35,0.01}{\textbf{\textit{#1}}}}
\newcommand{\WarningTok}[1]{\textcolor[rgb]{0.56,0.35,0.01}{\textbf{\textit{#1}}}}
\newcommand{\AlertTok}[1]{\textcolor[rgb]{0.94,0.16,0.16}{#1}}
\newcommand{\ErrorTok}[1]{\textcolor[rgb]{0.64,0.00,0.00}{\textbf{#1}}}
\newcommand{\NormalTok}[1]{#1}
\usepackage{graphicx,grffile}
\makeatletter
\def\maxwidth{\ifdim\Gin@nat@width>\linewidth\linewidth\else\Gin@nat@width\fi}
\def\maxheight{\ifdim\Gin@nat@height>\textheight\textheight\else\Gin@nat@height\fi}
\makeatother
% Scale images if necessary, so that they will not overflow the page
% margins by default, and it is still possible to overwrite the defaults
% using explicit options in \includegraphics[width, height, ...]{}
\setkeys{Gin}{width=\maxwidth,height=\maxheight,keepaspectratio}
\IfFileExists{parskip.sty}{%
\usepackage{parskip}
}{% else
\setlength{\parindent}{0pt}
\setlength{\parskip}{6pt plus 2pt minus 1pt}
}
\setlength{\emergencystretch}{3em}  % prevent overfull lines
\providecommand{\tightlist}{%
  \setlength{\itemsep}{0pt}\setlength{\parskip}{0pt}}
\setcounter{secnumdepth}{0}
% Redefines (sub)paragraphs to behave more like sections
\ifx\paragraph\undefined\else
\let\oldparagraph\paragraph
\renewcommand{\paragraph}[1]{\oldparagraph{#1}\mbox{}}
\fi
\ifx\subparagraph\undefined\else
\let\oldsubparagraph\subparagraph
\renewcommand{\subparagraph}[1]{\oldsubparagraph{#1}\mbox{}}
\fi

%%% Use protect on footnotes to avoid problems with footnotes in titles
\let\rmarkdownfootnote\footnote%
\def\footnote{\protect\rmarkdownfootnote}

%%% Change title format to be more compact
\usepackage{titling}

% Create subtitle command for use in maketitle
\newcommand{\subtitle}[1]{
  \posttitle{
    \begin{center}\large#1\end{center}
    }
}

\setlength{\droptitle}{-2em}
  \title{Teoria de Resposta ao Item: E-Class}
  \pretitle{\vspace{\droptitle}\centering\huge}
  \posttitle{\par}
  \author{}
  \preauthor{}\postauthor{}
  \date{}
  \predate{}\postdate{}


\begin{document}
\maketitle

\begin{Shaded}
\begin{Highlighting}[]
\KeywordTok{library}\NormalTok{(tidyverse)}
\KeywordTok{library}\NormalTok{(mirt)}
\KeywordTok{library}\NormalTok{(knitr)}
\end{Highlighting}
\end{Shaded}

\section{Organização dos dados}\label{organizacao-dos-dados}

\begin{Shaded}
\begin{Highlighting}[]
\NormalTok{ec <-}\StringTok{ }\NormalTok{seleclass }\OperatorTok
\StringTok{  }\KeywordTok{select}\NormalTok{(}\KeywordTok{contains}\NormalTok{(}\StringTok{"A"}\NormalTok{)) }
\end{Highlighting}
\end{Shaded}

\section{Como são os dados}\label{como-sao-os-dados}

\begin{Shaded}
\begin{Highlighting}[]
\KeywordTok{glimpse}\NormalTok{(ec)}
\end{Highlighting}
\end{Shaded}

\begin{verbatim}
## Observations: 192
## Variables: 30
## $ 1A  <int> 4, 5, 5, 5, 5, 4, 5, 4, 4, 4, 5, 4, 5, 4, 5, 4, 5, 5, 3, 4...
## $ 2A  <int> 2, 5, 5, 5, 5, 5, 3, 4, 3, 4, 5, 4, 5, 5, 4, 3, 5, 5, 4, 4...
## $ 3A  <int> 3, 2, 1, 1, 2, 2, 2, 2, 2, 2, 4, 3, 2, 4, 1, 4, 4, 2, 2, 2...
## $ 4A  <int> 4, 3, 5, 5, 3, 4, 2, 1, 4, 2, 2, 4, 1, 2, 2, 1, 2, 2, 2, 2...
## $ 5A  <int> 4, 4, 5, 5, 4, 2, 3, 4, 3, 4, 4, 5, 5, 4, 5, 3, 5, 4, 3, 4...
## $ 6A  <int> 2, 4, 5, 2, 4, 4, 3, 3, 2, 4, 4, 2, 5, 3, 4, 3, 3, 2, 4, 4...
## $ 7A  <int> 4, 1, 5, 1, 1, 1, 4, 3, 3, 3, 5, 2, 2, 3, 1, 1, 2, 1, 1, 1...
## $ 8A  <int> 4, 4, 5, 5, 4, 4, 4, 5, 4, 4, 5, 4, 5, 4, 4, 2, 5, 5, 4, 4...
## $ 9A  <int> 4, 4, 5, 3, 5, 5, 3, 4, 5, 3, 2, 4, 4, 2, 5, 3, 5, 1, 3, 4...
## $ 10A <int> 2, 4, 4, 3, 3, 3, 2, 4, 2, 4, 3, 4, 1, 2, 5, 2, 2, 5, 4, 4...
## $ 11A <int> 5, 5, 5, 5, 5, 5, 5, 5, 5, 4, 5, 5, 5, 4, 5, 1, 5, 5, 5, 5...
## $ 12A <int> 1, 4, 1, 1, 2, 2, 3, 2, 2, 4, 3, 4, 5, 2, 1, 2, 1, 1, 1, 2...
## $ 13A <int> 4, 5, 5, 5, 3, 5, 2, 4, 5, 4, 3, 4, 5, 4, 4, 5, 5, 3, 4, 4...
## $ 14A <int> 2, 3, 1, 4, 4, 4, 2, 5, 2, 3, 4, 2, 5, 3, 4, 4, 5, 4, 3, 4...
## $ 15A <int> 4, 3, 5, 4, 4, 3, 1, 1, 3, 4, 4, 4, 3, 4, 4, 3, 4, 4, 4, 4...
## $ 16A <int> 3, 2, 1, 3, 1, 4, 4, 1, 1, 4, 2, 2, 5, 1, 3, 2, 2, 2, 1, 2...
## $ 17A <int> 4, 2, 5, 5, 4, 2, 5, 5, 5, 4, 3, 4, 5, 3, 5, 4, 4, 2, 2, 4...
## $ 18A <int> 4, 4, 4, 5, 4, 4, 3, 5, 2, 4, 3, 4, 5, 3, 5, 5, 5, 3, 5, 4...
## $ 19A <int> 4, 4, 1, 3, 4, 3, 4, 5, 3, 4, 3, 5, 5, 3, 5, 5, 1, 3, 5, 3...
## $ 20A <int> 2, 3, 1, 5, 5, 3, 5, 1, 4, 4, 1, 4, 4, 3, 5, 5, 1, 5, 3, 5...
## $ 21A <int> 4, 2, 1, 1, 1, 5, 2, 1, 5, 2, 1, 4, 1, 2, 2, 3, 1, 4, 4, 2...
## $ 22A <int> 4, 3, 5, 5, 5, 5, 4, 5, 3, 4, 5, 5, 4, 3, 4, 4, 5, 5, 4, 4...
## $ 23A <int> 3, 4, 5, 4, 4, 5, 2, 5, 4, 4, 3, 5, 5, 2, 4, 3, 4, 4, 4, 4...
## $ 24A <int> 2, 3, 5, 5, 5, 4, 2, 3, 5, 4, 3, 3, 3, 4, 5, 4, 1, 5, 5, 4...
## $ 25A <int> 1, 2, 1, 1, 1, 1, 1, 1, 1, 2, 3, 1, 5, 2, 1, 3, 1, 1, 1, 1...
## $ 26A <int> 4, 4, 5, 4, 5, 4, 4, 5, 5, 4, 4, 5, 5, 4, 5, 4, 5, 4, 5, 4...
## $ 27A <int> 2, 1, 1, 1, 2, 2, 3, 2, 4, 2, 5, 1, 1, 3, 2, 2, 1, 2, 1, 2...
## $ 28A <int> 3, 1, 1, 1, 1, 1, 4, 3, 3, 2, 3, 1, 4, 3, 1, 1, 4, 2, 1, 1...
## $ 29A <int> 4, 3, 1, 3, 1, 5, 4, 4, 5, 2, 5, 4, 3, 4, 3, 3, 5, 3, 2, 4...
## $ 30A <int> 4, 5, 5, 5, 5, 5, 5, 5, 3, 4, 5, 5, 3, 3, 5, 5, 5, 5, 5, 5...
\end{verbatim}

\section{Criação do modelo}\label{criacao-do-modelo}

\begin{Shaded}
\begin{Highlighting}[]
\NormalTok{ec.tri <-}\StringTok{ }\KeywordTok{mirt}\NormalTok{(ec, }\DecValTok{1}\NormalTok{, }\DataTypeTok{itemtype =} \StringTok{'graded'}\NormalTok{, }\DataTypeTok{verbose=}\OtherTok{FALSE}\NormalTok{)}
\end{Highlighting}
\end{Shaded}

\begin{Shaded}
\begin{Highlighting}[]
\KeywordTok{summary}\NormalTok{(ec.tri)}
\end{Highlighting}
\end{Shaded}

\begin{verbatim}
##          F1      h2
## 1A   0.5983 0.35791
## 2A   0.3247 0.10544
## 3A  -0.1973 0.03895
## 4A   0.1478 0.02184
## 5A   0.3412 0.11639
## 6A   0.1868 0.03490
## 7A  -0.5052 0.25526
## 8A   0.5209 0.27138
## 9A   0.3911 0.15297
## 10A  0.4662 0.21736
## 11A  0.2647 0.07006
## 12A -0.5066 0.25661
## 13A  0.2184 0.04771
## 14A  0.4006 0.16049
## 15A  0.3612 0.13050
## 16A -0.0751 0.00564
## 17A  0.0611 0.00373
## 18A  0.4264 0.18181
## 19A  0.1997 0.03988
## 20A  0.3491 0.12186
## 21A -0.1781 0.03171
## 22A  0.5246 0.27522
## 23A  0.4359 0.19003
## 24A  0.3209 0.10300
## 25A -0.3885 0.15094
## 26A  0.4006 0.16047
## 27A -0.5446 0.29659
## 28A -0.5533 0.30615
## 29A -0.0936 0.00876
## 30A  0.6292 0.39590
## 
## SS loadings:  4.509 
## Proportion Var:  0.15 
## 
## Factor correlations: 
## 
##    F1
## F1  1
\end{verbatim}

\begin{Shaded}
\begin{Highlighting}[]
\KeywordTok{coef}\NormalTok{(ec.tri, }\DataTypeTok{simplify =} \OtherTok{TRUE}\NormalTok{)}
\end{Highlighting}
\end{Shaded}

\begin{verbatim}
## $items
##         a1     d1     d2     d3     d4     d5     d6
## 1A   1.271  5.993  4.145  2.925 -0.082     NA     NA
## 2A   0.584  2.527  1.254 -0.760     NA     NA     NA
## 3A  -0.343  1.359 -0.497 -1.175 -3.075     NA     NA
## 4A   0.254  2.027  0.767 -0.131 -1.407     NA     NA
## 5A   0.618  2.718  2.058  0.714 -1.100     NA     NA
## 6A   0.324  3.896  1.204 -0.300 -2.242     NA     NA
## 7A  -0.996  0.763 -0.899 -1.699 -2.724     NA     NA
## 8A   1.039  5.078  3.738  2.190 -0.396     NA     NA
## 9A   0.723  5.498  2.603  1.198  0.049 -1.459     NA
## 10A  0.897  4.004  2.137  0.488 -1.450     NA     NA
## 11A  0.467  5.361  3.733  3.387  2.083     NA     NA
## 12A -1.000  0.211 -1.760 -2.521 -4.293     NA     NA
## 13A  0.381  5.313  3.911  2.045  1.101 -0.436     NA
## 14A  0.744  3.887  1.712  0.108 -1.648     NA     NA
## 15A  0.659  3.441  2.473  0.565 -1.029     NA     NA
## 16A -0.128  5.261  1.213 -0.260 -1.083 -2.409     NA
## 17A  0.104  3.440  1.375  0.655 -0.644     NA     NA
## 18A  0.802  3.734  1.995  0.226     NA     NA     NA
## 19A  0.347  3.333  2.600  1.264 -0.177     NA     NA
## 20A  0.634  2.316  1.251  0.370 -1.011     NA     NA
## 21A -0.308  1.270  0.043 -0.733 -2.499     NA     NA
## 22A  1.049  2.803  0.125     NA     NA     NA     NA
## 23A  0.824  4.869  3.165  1.432 -0.955     NA     NA
## 24A  0.577  5.414  2.622  1.594  0.776 -0.367 -5.421
## 25A -0.718  4.816 -0.025 -1.421 -2.393 -4.077     NA
## 26A  0.744  5.528  4.417  2.955  0.279     NA     NA
## 27A -1.105  0.999 -1.087 -2.287 -3.966     NA     NA
## 28A -1.131 -0.591 -2.192 -3.082 -4.006     NA     NA
## 29A -0.160  2.910  0.986 -0.032 -1.622     NA     NA
## 30A  1.378  5.416  5.003  3.457  1.637     NA     NA
## 
## $means
## F1 
##  0 
## 
## $cov
##    F1
## F1  1
\end{verbatim}

\begin{Shaded}
\begin{Highlighting}[]
\NormalTok{p1 <-}\StringTok{ }\KeywordTok{plot}\NormalTok{(ec.tri, }\DataTypeTok{type=}\StringTok{"itemscore"}\NormalTok{)}
\NormalTok{p1}
\end{Highlighting}
\end{Shaded}

\includegraphics{tri_files/figure-latex/unnamed-chunk-14-1.pdf}

\begin{Shaded}
\begin{Highlighting}[]
\NormalTok{p2 <-}\StringTok{ }\KeywordTok{plot}\NormalTok{(ec.tri, }\DataTypeTok{type=}\StringTok{"info"}\NormalTok{)}
\NormalTok{p2}
\end{Highlighting}
\end{Shaded}

\includegraphics{tri_files/figure-latex/unnamed-chunk-14-2.pdf}

\begin{Shaded}
\begin{Highlighting}[]
\NormalTok{p3 <-}\StringTok{ }\KeywordTok{plot}\NormalTok{(ec.tri, }\DataTypeTok{type=}\StringTok{"trace"}\NormalTok{)}
\NormalTok{p3}
\end{Highlighting}
\end{Shaded}

\includegraphics{tri_files/figure-latex/unnamed-chunk-14-3.pdf}

\section{Selecionando apenas os item com boa disciminação e carga
fatorial
alta}\label{selecionando-apenas-os-item-com-boa-disciminacao-e-carga-fatorial-alta}

Cargas (loads)

1, 7,8, 22, 10, 22, 30

Melhor descriminação

1\emph{, 2, 5, 8}, 9, 10, 14, 15, 18, 20, 22\emph{, 23, 24, 26, 30}

Selecionados

c(1,8,22,30)

\begin{Shaded}
\begin{Highlighting}[]
\NormalTok{vars <-}\StringTok{ }\KeywordTok{c}\NormalTok{(}\StringTok{"1A"}\NormalTok{,}\StringTok{"8A"}\NormalTok{,}\StringTok{"22A"}\NormalTok{,}\StringTok{"30A"}\NormalTok{)}
\NormalTok{ecs <-}\StringTok{ }\NormalTok{seleclass }\OperatorTok
\StringTok{  }\KeywordTok{select}\NormalTok{(}\KeywordTok{one_of}\NormalTok{(vars)) }
\end{Highlighting}
\end{Shaded}

\section{Como são os dados}\label{como-sao-os-dados-1}

\begin{Shaded}
\begin{Highlighting}[]
\KeywordTok{glimpse}\NormalTok{(ecs)}
\end{Highlighting}
\end{Shaded}

\begin{verbatim}
## Observations: 192
## Variables: 4
## $ 1A  <int> 4, 5, 5, 5, 5, 4, 5, 4, 4, 4, 5, 4, 5, 4, 5, 4, 5, 5, 3, 4...
## $ 8A  <int> 4, 4, 5, 5, 4, 4, 4, 5, 4, 4, 5, 4, 5, 4, 4, 2, 5, 5, 4, 4...
## $ 22A <int> 4, 3, 5, 5, 5, 5, 4, 5, 3, 4, 5, 5, 4, 3, 4, 4, 5, 5, 4, 4...
## $ 30A <int> 4, 5, 5, 5, 5, 5, 5, 5, 3, 4, 5, 5, 3, 3, 5, 5, 5, 5, 5, 5...
\end{verbatim}

\section{Criação do modelo}\label{criacao-do-modelo-1}

\begin{Shaded}
\begin{Highlighting}[]
\NormalTok{ec.tris <-}\StringTok{ }\KeywordTok{mirt}\NormalTok{(ecs, }\DecValTok{1}\NormalTok{, }\DataTypeTok{itemtype =} \StringTok{'graded'}\NormalTok{, }\DataTypeTok{verbose=}\OtherTok{FALSE}\NormalTok{)}
\end{Highlighting}
\end{Shaded}

\begin{Shaded}
\begin{Highlighting}[]
\KeywordTok{summary}\NormalTok{(ec.tris)}
\end{Highlighting}
\end{Shaded}

\begin{verbatim}
##        F1    h2
## 1A  0.595 0.354
## 8A  0.488 0.238
## 22A 0.703 0.494
## 30A 0.694 0.482
## 
## SS loadings:  1.569 
## Proportion Var:  0.392 
## 
## Factor correlations: 
## 
##    F1
## F1  1
\end{verbatim}

\begin{Shaded}
\begin{Highlighting}[]
\KeywordTok{coef}\NormalTok{(ec.tris, }\DataTypeTok{simplify =} \OtherTok{TRUE}\NormalTok{)}
\end{Highlighting}
\end{Shaded}

\begin{verbatim}
## $items
##        a1    d1    d2    d3     d4
## 1A  1.261 6.009 4.142 2.927 -0.074
## 8A  0.952 4.994 3.681 2.174 -0.374
## 22A 1.682 3.364 0.159    NA     NA
## 30A 1.643 5.825 5.402 3.745  1.767
## 
## $means
## F1 
##  0 
## 
## $cov
##    F1
## F1  1
\end{verbatim}

\begin{Shaded}
\begin{Highlighting}[]
\NormalTok{p1 <-}\StringTok{ }\KeywordTok{plot}\NormalTok{(ec.tris, }\DataTypeTok{type=}\StringTok{"itemscore"}\NormalTok{)}
\NormalTok{p1}
\end{Highlighting}
\end{Shaded}

\includegraphics{tri_files/figure-latex/unnamed-chunk-18-1.pdf}

\begin{Shaded}
\begin{Highlighting}[]
\NormalTok{p2 <-}\StringTok{ }\KeywordTok{plot}\NormalTok{(ec.tris, }\DataTypeTok{type=}\StringTok{"info"}\NormalTok{)}
\NormalTok{p2}
\end{Highlighting}
\end{Shaded}

\includegraphics{tri_files/figure-latex/unnamed-chunk-18-2.pdf}

\begin{Shaded}
\begin{Highlighting}[]
\NormalTok{p3 <-}\StringTok{ }\KeywordTok{plot}\NormalTok{(ec.tris, }\DataTypeTok{type=}\StringTok{"trace"}\NormalTok{)}
\NormalTok{p3}
\end{Highlighting}
\end{Shaded}

\includegraphics{tri_files/figure-latex/unnamed-chunk-18-3.pdf}

\begin{Shaded}
\begin{Highlighting}[]
\NormalTok{sc <-}\StringTok{ }\KeywordTok{fscores}\NormalTok{(ec.tris)}
\KeywordTok{histogram}\NormalTok{(sc)}
\end{Highlighting}
\end{Shaded}

\includegraphics{tri_files/figure-latex/unnamed-chunk-18-4.pdf}


\end{document}
